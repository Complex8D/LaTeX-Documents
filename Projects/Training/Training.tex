\documentclass{article}
\usepackage[utf8]{inputenc}
\usepackage{amsmath}

\title{Träningsplanering}
\date{15-02-2019}
\author{Oscar Bohlin}

\begin{document}

\pagenumbering{gobble}
\maketitle
\newpage
\pagenumbering{arabic}
\tableofcontents
\newpage

\section{S.M.A.R.T}

\subsection{Specifikt}
	
	Springa 1-4 km 2 - 4 gånger i veckan, max äta 200g godis och snacks per vecka, äta samtlig mat enligt tallriksmodellen, äta gröt till frukost och få ett bättre sov-schema. Med bättre sov-schema menas att jag ska försöka sova 8 timmar varje natt och göra så att skillnaden mellan tiderna jag går lägger mig är så liten som möjligt.

\subsection{Mätbart}
	Det som kan mätas är hur min kondition förändras, det vill säga vilken tid jag springer $X$ kilometer med vilken puls. Antal timmar sömn är också mätbart. Däremot kan inte målet med tallriksmodellen mätas. Jag har en smart klocka som mäter i princip allt inom ett träningspass och hur mycket jag sover varje natt och delar upp det i vanlig och djup sömn så jag ser hur bra jag sover. Den förbättrade konditionen ska synas på bip testet vi kommer att göra.

\subsection{Attraktivt}
	Jag vill kunna gå upp för en trappa utan att bli andfådd, eller att bli trött efter en kort promenad. Jag vill även förbättra min psykiska hälsa. Jag vill få en bättre hållning och kunna koncentrera mig under en längre period. 

\subsection{Realistiskt}
	Jag skall kunna springa 4 kilometer på 20 minuter, därmed en medelhastighet på $ v_{medel} = 12 \frac{km}{h}$ eller $ v_{medel} = \frac{12\frac{km}{h}}{3.6} = 3.33 \frac{m}{s}$

\subsection{Tidsbestämt}
Torsdagen den 28 Mars, 2019 ska jag kunna springa 4 kilometer på 20 minuter.\\

\section{Förtydligande}
	Mitt mål relaterat till kost är följande: Jag ska max äta 200g godis och snacks varje vecka och jag ska äta måltiderna enligt tallriksmodellen. Till frukost ska jag äta gröt så jag håller mig mätt och blodsockerhalten varierar inte lika mycket så jag kan behålla fokus. \\

	Mitt mål relaterat till sömn är följande: Mitt mål är att sova 9 timmar varje natt. Det innebär att jag vare sig ska sova mindre eller mer. Tiden jag går och lägger mig ska vara ungefär samma samtliga dagar, samma princip för tiden jag vaknar. Detta mål kommer vara svårt att följa då jag har stora schema skillnader på första lektionen. \\

	Mitt mål relaterat till träning är följande: Jag ska träna enligt grov- och detaljplaneringen med målet att springa 4 kilometer på 20 minuter.

\section{Grovplanering}

\subsection{Vecka 7}
	Introduktionsvecka, börja med sömnschemat. Börja med tallriksmodellen och äta gröt till frukost. Jag ska ut och kolla hur det går att springa 3 kilometer. I slutet av veckan ska jag gå på promenader för övning och uppfriskning. 

\subsection{Vecka 8}
	Inget speciellt denna vecka. Jag ska börja veckan med fortsatta promenader för i mitten av veckan börja springa/jogga. Mat och sömnplaneringen fortsätter som vanligt. Jag ska springa 3.5 kilometer två gånger i långsamt tempo, snabbare än 25 minuter.

\subsection{Vecka 9, Sportlov}

	Sportlovsvecka, jag åker bort, sömn och matschema kommer genomgå stora förändringar. Planerat är bland annat två dagars utförsåkning och där utöver lätta promenader. För att kompensera för den förlorade träningen kommer jag istället att göra enkla fysövningar. Armhävningar och plankan. I slutet av veckan kommer jag eventuallt gå på en promenad på 5 km, beroende på hur trött jag är efter hemresan. 
		
	
\subsection{Vecka 10}

	Återhämtning efter sportlovet, börja med promenader och sedan fortsätta springa 3.5 kilometer. Återuppta sömn- och matplaneringen. Börja med intervallpass. 

\subsection{Vecka 11}

	Fortsätta som vanligt, fortsätta med intervaller, springa 4 km och 3.5 km. Ungefär halvvägs så kan kontrollera på fredagen om jag har en chans att uppnå mitt mål. Även gå promenader i snabbt tempo.

\subsection{Vecka 12}

Samma uppsättning som \textbf{{Vecka 11}} med samma målsättning och tider. 4 och 3.5 km samt intervaller och någon promenad. Sömn och mat planeringen fortsätter som vanligt. 

\subsection{Vecka 13}

	Framförallt satsa på 4 km distans med någon 3.5 km tur i snabbt tempo, snabba promenader. Torsdagen skall jag se om jag har uppnått mitt mål. 


\section{Detaljplanering}
	Detaljplaneringen finns bifogad i en Excel fil.
	

\end{document}
