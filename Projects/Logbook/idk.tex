\documentclass[a4paper, 12pt]{article}
\usepackage[utf8]{inputenc}
\usepackage{amsmath}
\usepackage[swedish]{babel}
\usepackage{lmodern}
\usepackage[parfill]{parskip}
\usepackage[pdftex,
            pdfauthor={Oscar Bohlin},
            pdftitle={Träningslogg}]{hyperref}


\hypersetup{pdfborder = {0 0 0}}
\title{Träningslogg}
\date{15-02-2019 - 28-04-2019}
\author{Oscar Bohlin}

\begin{document}

\pagenumbering{gobble}
\maketitle
\newpage
\pagenumbering{arabic}
\tableofcontents
\newpage


\section{Vecka 7}

\subsection{Torsdag 14-02-2019}

    Referens löpning för uppsättning av det realistiska målet. Jag sprang 3.5 km på 25 minuter, vilket ger en medelhastighet $v = \frac{3.5km}{\frac{25min}{60min}} = 8.4\frac{km}{h}$ eller $\frac{8.4\frac{km}{h}}{3.6} = 2.33\frac{m}{s}$. Medelpulsen $p_{medel} = 165$ och maxpulsen $p_{max} = 194$

    Jag kände jag mig inte så trött under passet, jag tror därmed att jag kunde ha sprungit längre eftersom;

\begin{enumerate}
     \item Jag stannade en gång för att hälsa på en kompis och grafen visar att min medelhastighet ökade bastant efter hälsandet.
    \item Jag sprang över ett isfält. 

\end{enumerate}

\subsection{Fredagen 15-02-2019}

    Vilodag 

\subsection{Lördagen 16-02-2019}
    Tennisen var inställd så jag gick en promenad istället, 5 km på tiden 1 timme. 

\subsection{Söndag 17-02-2019}
    Jag gick enligt planeringen en promenad på 5 km på tiden 45 minuter. Medelpulsen $p_{medel} = 165\frac{slag}{minut}$.

\subsection{Sammanfattning Vecka 7}

    Reflektion: Träningsplanering går som denq ska utan några problem. Målet med minskat godis- och snacksintag är också enligt planering. Däremot har vare sig sömn- eller matmålet gått så bra.
    
    Gällande sömnen så sover jag för lite, inte med en stor del utan varierar från 3-1 timme, största orsaken är troligast att jag använder mobilen strax innan jag går och lägger mig. Fortsättningsvis ska jag läsa strax innan jag går och lägger mig. 
    
    Angående matmålet så ligger problemet hos grönsakerna, jag äter för lite. Lösningen är enkel och problemet måste fixas omgående. 



\section{Vecka 8}

\subsection{Måndag 18-02-2019}
    Vilodag 
    

\subsection{Tisdag 19-02-2019}

    Enligt detaljplaneringen skulle jag ha springit 3.5 km på mindre än 25 minuter.
    
    Jag sprang 3.05 km på 23 Sßminuter och 8 sekunder. Medelhastigheten $v_{medel} = 9.1\frac{km}{h}$ vilket är tydlig ökning sedan min referenslöpning på $v_{medel_{referens}} = 8.4\frac{km}{h}$. Medelpulsen $p_{medel} = 175$ och $p_{max} = 198$.
    
    Min puls ökade med ca 10 $\frac{slag}{minut}$ vilket går i fel riktning. Detta beror troligast på att jag ansträngde mig lite mer och mötte mer vindmotstånd denna gång. 

\subsection{Onsdag 20-02-2019}

    Vilodag, gick en extrem kort promenad och tog totalt 11200 steg.
    
\subsection{Torsdag 21-02-2019}

    Min träningsklocka strulade väldigt mycekt idag. Det ledde till att jag inte fick fulla träningspasset kartlagt. Däremot sprang jag 4km på mindre än 25 minuter.


\subsection{Fredag 22-02-2019}

    Vilodag, ingen spacifik fysisk aktivitet. 

\subsection{Lördag 23-02-2019}

    Vilodag.

\subsection{Söndag 24-02-2019}

    Ingen stor fysisk aktivitet. Kunde ej utföra detaljplaneringen. 

\subsection{Sammanfattning Vecka 8}

    Jag anser veckan har gått väldigt bra. Jag utförde träningsplaneringen, till och med bättre resultat än förväntat så jag kunde springa längre än planerat och på en kortare tid. 
    
    Sömnplaneringen går fortfarande inte jättebra, jag sänkte målet till 8 $\frac{h}{natt}$.
    
    Matmålet går bättre, jag har implementerat metoden i skolan men har fortfarade mycket att jobba med det hemma.
    
    Snacksmålet överstred inte gränsen. 


\section{Vecka 9, Sportlov}

\subsection{Måndag 25-02-2019}

    Enligt planering vilodag. Jag åkte skidor.
    
\subsection{Tisdag 26-02-2019}

    Utförde enligt planeringen armhävningar, situps och plankan så länge jag orkade. 

\subsection{Onsdag 27-02-2019}

    Vilodag, gick en lätt promenad

\subsection{Torsdag 28-02-2019}

    Utförde enligt planeringen fysövningarna samt gick en kort promenad.

\subsection{Fredag 01-03-2019}

    Vilodag.
    
    Gick en lång promenad.

\subsection{Lördag 02-03-2019}

    Vilodag. Satt i ne bil hela dagen, hemfärd från Umeå. 

\subsection{Söndag 03-03-2019}

    Detaljplaneringen sade 5 km promenad men utfördes ej, mina ursäkter är följande: 

    \begin{enumerate}
        \item Det var kallt och blåsigt, inget trevligt promeneringsväder.
        \item Jag var extremt trött från gårdagens bilfärd.
        \item Jag hade andra saker att fixa.
        
    \end{enumerate}

\subsection{Sammanfattning Vecka 9, Sportlov}

    Veckan har gått dåligt. Sömnplaneringen uppfördes sporadiskt, planen att äta gröt till frukost utförden inte några dagar, tallriksmodellplanen var också väldigt svår att utföra då, jag försökte så gott jag kunde.
    
    Träningsplaneringen utfördes, tyvärr krockade den lite med skidåkning et cetera vilket påverkade resultatet men utfördes utan problem. Däremot gick jag inte promenaden på söndag men det borde inte ha en jättestor påverkan. 

\section{Vecka 10}

\subsection{Måndag 04-03-2019}

    Vilodag.

\subsection{Tisdag 05-03-2019}

    På grund av den kalla temperaturen utomhus kunde jag inte springa min vanliga sträcka utan sprang på löpbandet. Jag sprang 3.0 km på ca 21 minuter. Medelpulsen $p_{medel} = 168\frac{slag}{minut}$ och $p_{max} = 198\frac{slag}{minut}$.
    
    Jag ser en positiv effekt än så länge, vilket förvisso kan bero på att jag sprang på ett band och inte i naturen men min medelpuls har sänkt något, även om maxpulsen är den samma, vilket skulle kunne tolkas som att jag når min fysiska maxpuls. Formeln för maxuls brukas skrivas som $f(x) = 220 - x$, $f(17) = 203$ vilket antyder att jag kan ha kommit upp i min fysiska maxpuls. 

\subsection{Onsdag 06-03-2019}

    Vilodag.

\subsection{Torsdag 07-03-2019}

    Jag sprang enligt träningsplaneringen. Jag upplevde ett större motstånd från benen som jag inte känt tidigare, detta kan bero på att jag var ganska trött redan innan träningspasset i kombination med att jag åt dåligt på lunchen. 

    Hursomhelst så sprang jag 3.52 km på 21 minuter och 12 sekunder. Medelhastigheten $V_{medel} = 10\frac{km}{h}$ med en maxhastighet på $V_{max} = 11.4\frac{km}{h}$. 
    
    Det jag förut antog om min maxpuls visade sig vara fel då jag uppnådde en maxpuls på $p_{max} = 203\frac{slag}{minut}$ vilket enligt formeln $f(x)$ är min exakta maxpuls. Min medelpuls $p_{medel} = 181\frac{slag}{minut}$ vilket motstrider det jag sade i tisdags, att min medelpuls har minskat. 



\end{document}

